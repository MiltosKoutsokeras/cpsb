\chapter{Programming concepts}

To program a CPS-1 board mean to produce four sets of ROMs. These are the graphics assets holding the sprites and tilemap contained in the GFX ROM, the 68000 instructions to pilot the control system, Z-80 ROM which contains not only instructions but also YM2151 bytecode to generate the FM music, and finally MSM6295 ROM where are stored the audio samples.

 \begin{figure}[H]
\sdraw{1.0}{cps1_arch}
\caption*{Three parts, two ROMS}
\end{figure}

To make things even more difficult each of these sets use a specific data width and interleaving according to what was described in the hardware chapter.

\pagebreak
% Along the way must be decided of communication protocols for each interfaces used by each system to communicate. Namely the GFX RAM where the control system talks with the gfx system and the two 1-byte latches where control send commands to the audio system.


\section{Language}
For all their CPS-1 titles, Capcom developers used z80/m68k assembly. They did not have much choice anyway since compilers were expensive in the late 80s and not that good at producing compact code. As we will see, ROM space was precious and controlling the volume of instructions with accuracy was paramount.

Since the goal of this chapter is to explains how things works, it uses C for its greater readability and wider knowledge base among programmer. A little bit of assembly is used but only to bootstrap the CPUs.

If you ever decide to take your CPS-1 program to the next level, you will without a doubt have to go the assembly way.

\section{Down to the metal}
In a programming world were developers are more and more removed from the hardware, programming the Z-80 or the M68000 is a joy.

Without libraries, frameworks, dynamic linker, syscalls, virtual memory, and loader, development happens down to the metal.

An innocent three-line C program is enough to get a glimpse of what awaits us.

\lstinputlisting[language=C]{src/code/variablesDeclaration.c}

After compilation and linking, this program will result into a \icode{program.rom} which will be burned into EPROMs which are mapped somewhere in the CPU address space.

\nbdraw{prog_rom_mapping}

Variable \icode{varA} which is uninitialized, readable, and writable. The linker will have assigned the first location in RAM \icode{B}. Any code that reads or write \icode{varA} in our C program will result in instructions addressing \icode{B}. Likewise, reference to \icode{varB} will result in instructions manipulating \icode{B+1}.

Variable  \icode{varC} is initialized, readable, but NOT writable. The linker will have assigned offset \icode{0x0} in \icode{program.rom} and set it to the desired value (\icode{6}. Knowing that ROM mapping start at \icode{A}, any code referring to \icode{varC} result in instructions manipulating \icode{A}. Likewise, variable \icode{varD} is placed in \icode{program.rom} at offset 1, where value \icode{38} is stored. Instructions manipulates address \icode{A+1}.

Finally we come to variable \icode{varE}. Since it is writable, the linker will have used the next available address in RAM and used \icode{B+2}. But how can the linker initialize that location since it can only write to file \icode{program.rom} which is not mapped there? 

The answer is that it cannot. The "copy-down" must be done when the CPU starts, before executing the \icode{main} function of our program is called. 

This bootstrap, called \icode{crt0}, and other subtle mecanims are detailed in the CPU sections.


\section{Interrupts}

All APIs and interfaces of the machine are well defined. The GFX system expects commands via the GFX RAM and the 68000 bus arbitrates read/write operations. On the Audio side, the YM2151 exposes two registers (cmd \& data) and so does the MSM6295 both access are arbitrated by the memory map. The on area which is undefined is the interface between the Control and Audio systems.

Try to think of a design yourself. You have two 1 byte latches, a 68000 running at 10MHz which can write in them but not read. On the other end is a Z-80, working at 3.579 MHz which can read the latches but not write them. How do you make these two CPUs talk to each other with 100\% reliability?

There are many ways to solve this problem. Here is how Street Fighter II board solved it. The first issue is that the reader runs slower than the writer which makes it possible to "miss" a command. Inverting the ratio is done via CPU interrupts. The 68000's IPL1 line is directly connected to the VSYNC line of the video system. Likewise, the Z-80 INT line is connected to the timer (CT1) line of the YM2151.

\nbdraw{interrupt_snd}

This way the writer ticks every 16ms while the reader ticks every 4ms. 

\nbdraw{interrupt_ctrl}

This ensures no commands can be dropped but introduces the problem of duplicates. To avoid these, the Z-80 disregards an input if it is the same as the last one.

This introduces an ultimate problem if the 68000 needs to send the same command twice in a row. To work around this, the writer commits on never writing the same byte twice which is done via a no-op byte (0xFF) written after every command.


\begin{trivia}
In the protocol we described, only one latch is used. The second one is completely disregarded. This is exactly the way Street Fighter II board operates	.
\end{trivia}

\subsection{Implementation details}
Leveraging interrupt turns our single threaded CPU into dual threaded system (except one has to restart from the beginning every time) able to reliably exchange values. But we still have to decide of what meaning give ot the data. 


Many approach work. A quick and dirty implementation could use a single byte and divide its value space in two. A MSB set to one \icode{0x80} request a sound effect playback and a MSB set to zero \icode{0x00} indicates a music playback. This leaves 126 sound values and 126 music values. On the Z80 side, volume can be hard-coded, sample playback and use round-robin between two channels and the music track can use the two other channels to embellish YM2151 melodies. 

On Street Fighter 2, developers used a circular buffers where value are staged	.

The interrupt thread consumes the buffer and, depending on which side of the latches they are, either read or write one byte.

\nbdraw{latches_intercafe}

\nbdraw{latches_uml}

On Final Fight, there is no translation table.

\pagebreak


\section{Timers}
The interrupt system we just saw is instrumental to keep track of wall-time (the time perceived by players). 

On the Control side it is useful to sample user input at an appropriate interval and make sure GFX animation are player at the right speed. As we saw before the VSYNC interrupt allows to keep track of things in roughly 16ms interval.

Sound also needs to keep track of wall time in order to feed YM2151 music instructions at the right pace. Here we have more flexibility. Whereas VSYNC frequency is fixed,  the Z-80 reprogram the YM2151 timer after each trigger. For the example we will see there is no need to do that so our timer progresses in 4ms increments.

\section{Randomness}
Generating a series of pseudo-random numbers can be achieved with a Maximum-Length LFSRs (Linear Feedback Shift Register) and a proper seed. The issue is what to use for the seed. We could use a timer keeper and use the first player inputs. But that would open our games to players gaming the system. A better seed source is to use uptime.

\section{CCPS, the CPS-1 SDK}
Framework CCPS is the CPS-1 SDK companion of this book. It is free and opensource. You can get it with the following command.

\lstinputlisting[language=Bash]{src/code/clone_ccps.sh}