\chapter{Programming the CPS-1}

To program a CPS-1 board mean to produce five sets of ROMs. These are the graphics assets holdring the sprites and tilemap contained in the GFX ROM, the 68000 instructions to pilot the control system, Z-80 ROM which contains not only instructions but also YM2151 bytecode to generate the FM music, and finally MSM6295 ROM where are stored the audio samples.

\vfill
DRAWING of ALL SYSTEMS

Along the way must be decided of communication protocols for each interfaces used by each system to communicate. Namely the GFX RAM where the control system talks with the gfx system and the two 1-byte latches where control send commands to the audio system.

\pagebreak

\section{Languages}
Historially CPS-1 developers used assembly to program. It is well suited for the kind of low-level operations to perform but most importantly it is second to none when it comes to instructions compactness. As we will see ROM space is not as plentiful as it seems.

Since the objective of this chapter is to explains how things works, it uses C for its greater readability and wider knowledge base accross programmer.

If you ever decide to take your CPS-1 program to the next level, you will without a doubt have to go the assembly way. Just not today.


\section{Down to the metal}
In a programming world were developers are more and more removed from the hardware, programming the Z-80 or the M68000 is a joy.

Here, there are no libraries, no frameworks, no dynamic linking, no syscalls, no virtual memory, no linker, and no loader.

We are down to the metal which it a great opportunity to appreciate all the work operating systems do for us on a dialy basis.

An innocent three-line C program is enough to get a glimpse of what awaits us.

\lstinputlisting[language=C]{src/code/variablesDeclaration.c}

After compilation and linking, this program will result into a \icode{program.rom} which will be burned into an EPROM, itself mapped somewhere in the CPU address space.

DRAWING

Let's take a look at what the linker will have done with these three lines.
 For \icode{varA} which is uninitalized, readable, and writable, the linker will have assigned the first location in RAM \icode{B}. Any code that reads or write \icode{varA} in our C program will be using address \icode{B}.

\section{Programming the audio system}
Programming the Z-80 is a rewarding experience that takes a programmer down to the metal. The goal of the exercise if to produce a raw block of 64 KiB of instructions that will be burned into an EPROM from which the CPU will start execution at address \icode{0x0200}. To help in this endeavor, the compiler can place R/W variables and the stack in RAM which are mapped at \icode{0xD000} in processor address space. Needless to say there are no memory protection, process, or even thread here.

\nbdraw{interrupt_snd}

\section{Z-80 crt0}
To speak in concrete terms, let's take the example of a seemingly innocuous C code declaring three global variables.



Can you guess which address the compiler will use for each of these three variables?

The first easy case is varA which is uninitialized with read and write purposes. The compiler will naturally use address 0xD000. If the variable is read before it is written it will have whatever value the RAM had at this location. 

The second easy case if varB which is a const read-only. It can safely be placed around \icode{0x0200} in ROM since it will only be read.

But what should be done with varB? It can be read but it can also be written to so it should be in RAM (likely \icode{0xD001}) but it is initialized with a value coming from the ROM. An operating system would 

\section{Audio assets for the audio system}

\section{Programming the control system}

\nbdraw{interrupt_ctrl}

\section{68000 crt0}
\pagebreak
\simg{0.53}{x68000_XVI.png}

\section{GFX assets for the gfx system}

\pagebreak
\img{smc-70_ad.jpg}
\img{smc-70_capcom.png}