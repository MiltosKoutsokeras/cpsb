\chapter*{Forewords}

There used to be a time when video-games enthusiasts could only experimence the very best in places called "arcades". 

In the early 90s, home consoles such as the Super Nintendo, the Sega Genesis, or the NEC PC Engine were ramping up in term of horsepower but those were still a far cry from the hardware found in coin-ops.

These video-games featuring multitudes of huge sprites covering the whole screen, beautiful colors, digitized sounds, and engaging high quality musics were in a league of their own.

Accessing to arcades was an adventure in itself. Quarters had to be gathered, means of transportation acquired, and paper maps studied. Some carpooled while others used their bike. Lucky ones had "amusement venues" dedicated to video-games in their hometown while others found themselves in a dirty pub surrounded by adults whom did not seem to have much magic happening in their lives. 

The overhead was tremendous. Many hours were invested sometimes resulting in only a few minutes of play time. Despite all these obstacles, video-games connoisseurs found the attraction irresistible. Players of all ages, origins, and social backgrounds gravitated toward the same place in order to follow their passion. 

Rows of cabinets lined up made for a highly competitive environment. Publishers only had a fraction of a second to catch a player attention and most importantly their quarters. It is during that time that a young company named Capcom  managed to elevate itself above the competition and turn itself into an icon.

Producing seemingly one masterpieces after another, the genesis of Street Fighter II, Ghouls 'n Ghosts, or Final Fight belongs in history books. Unfortunately when I started being interested on the topic, I found little to satisfy my curiosity and next to noting about the engineering side of things. 

The fierce rivalry between publishers resulted in extreme secrecy. Artists, Programmers, and Game Designers were not credited with their name in order to avoid poaching. As for the hardware powering the software, nothing ever officially transpired except for a code name, \textbf{CP-System}.

This book attempts to shed some light over the mystery platform. It is an engineering love letter to the machine that enabled Capcom's tremendous success . 

It may be argued that knowing how something works takes away the magic. Looking past the smoke and mirrors does reveal the cables and gears moving the big head of Oz. But to learn how problems were solved though force of will, dedication, and ingenuity only command, at least for me, a greater appreciation for the craft.

As we are about to discover how the rabbit came out of the hat, I hope readers will enjoy the ride but more importantly I hope the story of Capcom's dedication will inspire them as much as it has inspired me. 

- Fabien Sanglard