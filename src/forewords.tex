\chapter*{Forewords}

There used to be a time when video-game enthusiasts could only experience the very best in places called "arcades". 

In the early 90s, 16-bit home consoles such as the Super Nintendo, the Sega Genesis, or the NEC PC Engine were ramping up in terms of horsepower. However, they were a far cry from the hardware found in coin-operated "Amusement Machines".

Nicknamed "coin-ops", these cabinets ran video-games featuring multitudes of huge sprites covering the whole screen, beautiful colors, digitized sounds, and engaging high quality music. These machines were in a league of their own.

Accessing arcades was an adventure in itself. Quarters had to be gathered, means of transportation acquired, and paper maps studied. Some carpooled while others used their bikes. Lucky ones had "amusement venues" dedicated to video-games in their hometown while others found themselves in a dirty pub surrounded by adults who did not seem to have much magic happening in their lives. 

Amount of play-time was directly correlated to skill level. Coins were spent carefully, after having studied other people's techniques. The only certainty resulting from the expedition was a day ending with empty pockets. 

Despite all these obstacles, video-game connoisseurs found the attraction irresistible. Players of all ages and origins gravitated to the same places in order to follow their passion. 

Rows of lined up cabinets created a highly competitive environment where publishers only had a few seconds to catch a player's attention and, most importantly, their quarters. It was during this time that a young company named Capcom  managed to rise above the competition, seemingly producing one masterpiece after another, and turn itself into an icon.

The history of Capcom and the genesis of Street Fighter II, Ghouls 'n Ghosts, and Final Fight belongs in history books. Unfortunately when I started researching the topic, I found little to satisfy my curiosity and next to nothing about the engineering side of things. 

The fierce rivalry between publishers warranted extreme secrecy. Artists, programmers, and designers were only credited with their nicknames in order to avoid poaching. As for the hardware powering Capcom's titles, nothing ever officially transpired except for a code name, \textbf{CP-System}.

This book attempts to shed some light over the mystery platform. It is an engineering love letter to the machine that enabled Capcom's tremendous success. 

- Fabien Sanglard\\
 Version 1 (September, 2022)
