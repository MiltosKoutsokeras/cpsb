\chapter{Forewords}

Some video-games are like magic tricks.

Faced with something seemingly impossible, some people will take away a fond memory but others will want to understand
how it works.

When acquired post-mortem, knowledge does not destroys an illusion or even lessen its experience. If anything it takes the amazement 
one step further. It commands even more respect towards the magician. It pays respect to the artist. Knowing, after the fact, is a celebration.  

Being in the shoes of the actor, even for a brief time, and even as a consumer, is like bonding with these people who gave us so much. Of course they did it for a living but some of them took it much further than their salary mandated. These resulted in \emph{prestige-ious} titles.

By understanding how hard it is to get it right, the countless details, the innumerable hours of practice, the imagination required, in one word, the dedication it took, we inevitably develop greater fondness for their craft.

Older video games such as the one experienced by enthusiasts in the arcades of the mid-90s were magic. They did something players could not experience at home. Players' hearts raced and eyes sparkled. They took us to another world. The experience was re-inforced by the effort it took to get access to arcade games. 

Players of this era did not fly to Vegas but they had to make numerous arrangement to get there. Quarters had to be gathered,  means of transportation acquired, and paper maps studied, often to land in a dirty bar surrounded by adults that had did not have much wizardry in their lives. But what a reward to see Capcom rivaling with SNK for our attention. 

As we are about to discover how the rabbit came out of the hat, I hope readers will enjoy the ride but more importantly I hope the story of the CPS-1 will inspire them as much as it inspired me.

In every aspect of our lives, by striving for excellence, by being considerate of others, and being giving we can make this world we share together a better place. 

We can all be magicians.

- Fabien Sanglard