\chapter*{Forewords}

There used to be a time when video-games enthusiasts could only experience the very best in places called "arcades". 

In the early 90s, home consoles such as the Super Nintendo, the Sega Genesis, or the NEC PC Engine were ramping up in term of horsepower but those were still a far cry from the hardware found in coin-operated "Amusement Machine".

Nicknamed "coin-ops", these cabinets ran video-games featuring multitudes of huge sprites covering the whole screen, beautiful colors, digitized sounds, and engaging high quality musics. These machines were in a league of their own.

Accessing to arcades was an adventure in itself. Quarters had to be gathered, means of transportation acquired, and paper maps studied. Some carpooled while others used their bike. Lucky ones had "amusement venues" dedicated to video-games in their hometown while others found themselves in a dirty pub surrounded by adults whom did not seem to have much magic happening in their lives. 

The amount of play-time was directly correlated to the skill level. Coins were spent carefully, after having studied other people's techniques. The only certainty resulting from the expedition was a day ending with empty pockets. 

Despite all these obstacles, video-games connoisseurs found the attraction irresistible. Players of all ages and origins gravitated toward the same place in order to follow their passion. 

Rows of cabinets lined up created a highly competitive environment where publishers only had a fraction of a second to catch a player attention and most importantly their quarters. It is during that time that a young company named Capcom  managed to elevate itself above the competition, seemingly producing one masterpieces after another, and turn itself into an icon.

The history of Capcom and the genesis of Street Fighter II, Ghouls 'n Ghosts, and Final Fight belongs in history books. Unfortunately when I started researching the topic, I found little to satisfy my curiosity and next to nothing about the engineering side of things. 

The fierce rivalry between publishers warranted extreme secrecy. Artists, programmers, and designers were only credited with their nickname in order to avoid poaching. As for the hardware powering the software, nothing ever officially transpired except for a code name, \textbf{CP-System}.

This book attempts to shed some light over the mystery platform. It is an engineering love letter to the machine that enabled Capcom's tremendous success . 

We are about to discover how the rabbit came out of the hat. Some may argue that looking past the smoke and mirrors takes away the magic by revealing the occasionally messy arrangement of cables and gears moving the big head of Oz. Chances are that if you are reading this book, learning the amount of hard work it took to create these amazing games will only command a greater appreciation for the craft.

ROUND 1......FIGHT!

- Fabien Sanglard