\chapter{GFX System}
Since the graphic pipeline of the CPS-1 is hard-coded in the silicon of the CPS-A and CPS-B, there is no code to write and nothing to compile. 

\begin{figure}[H]
\sdraw{1.0}{gfx_arch}
\caption*{The GFX System components}
\end{figure}

As explained in the hardware chapter, all assets use the same pixel format where four bytes encode eight pen values. But that would be an oversimplification to think that slicing PNG into tiles and placing them anywhere in the GFX ROM is enough.

The major constraint to consider is that GFXROM have "hard-coded regions" where OBJ, STAR1, STAR2, SCR1, SCR2, and SCR3 must reside. If a tile is not where it should, the draw command is simply ignored.

On the board we took as example in the hardware chapter, Street Fighter II, the \icode{STF29} PAL slice the GFXROM in an unique set of four areas. A board running Final Fight uses a \icode{S224B} PAL which also uses four areas. An PAL programmed for Forgotten Worlds, the \icode{LW621} is even more complex since it divide the GFXROM in six areas to also include STAR1 and STAR2 bytecode. Each game is unique. 

Keep all this in mind if you write your own ROM generator or use \icode{ccps} if you don't want to worry about it.

\section{Tile format}
All tiles are stored continuously in memory, using groups of four bytes encoding eight pens. The dimension of a tile varies depending on the layers. OBJ and SCR2 uses the same 16x16 tiles which mean their tiles are 16*16 / 2 = 128 bytes. For the fine tuned SCR1 which uses 8x8, each tile takes 32 bytes. Finally the larger SCR3 uses 512 bytes for each of its 32x32 tiles.

\section{GFX Layout}
Depending on the board targeted, the game may have up to 6MiB of GFX ROM available. The logical ROM is made of groups of four hardware chips sequencially arranged. On a board like Street Fighter, we find three groups of four. Inside each group, chips are interleaved every two bytes.

\red{DRAWING: HOW DOES IT WORK? 4*16 = 64-bit bus or 2x32?}

\nbdraw{build_graph_gfx}

\img{smc-70_ad.jpg}

\section{Back then...}
Thanks to artists interviews, many details of their process transpired.

\subsection{Following the paper trails}
If breaking free of the rectangular sprites was a blessing for the artists, it was a problem for Capcom project managers. In an era where ROM chips were very expensive, a game was allocated a ROM budget at its beginning which it could not exceed.

Before the CPS-1, remaining within the budget was a simple matter of a division. The number of sprites allowed to the art team was ROM size / rectangular sprite size. But the free form factor introduced a tracking problem.

\begin{q}{Akira Nishitani, SF2 Producer (Capcom)}
In order to make the best use of the capacity we had, we wrote the ROM’s capacity on a board, and cut and paste the pixel characters on the board.

If there was space left on the board, then there was open capacity in the ROM. So, from there we started filling in the spaces, like a puzzle.

One thing that happened that's kinda interesting, we saved making the ending for last, and by the time we got there we were all out of capacity. We were wondering what to do, when we found a board that had gone missing under a desk.

We called it the "Mirac-ulous Memory."
\end{q}



 \begin{figure}[H]
\img{rom_sheet_dhalsim.jpg}
\caption*{Dhalsim released paper sheet}
\end{figure}

For a game like Street Fighter II, a budget of 6MiB GFX was approved. With 4.6 MiB dedicated to sprites, 144 OBJ sheets were printer. That was a lot at the time and only warranted because the team had managed to score a huge hit with Final Fight on a tiny 2MiB budget. 

To this day, only two of these sheets have ever been released\cite{fdasd}\cite{fs}. If it will without a doubt dissapoint the most stubborn of us to know that Capcom may never released more, those people can rejoice. There is a way to look at them thanks to the imprint left in the GFXROM. 

With knowledge of the tile format and the pixel format, they can all be reconstructed.




 \begin{figure}[H]
\nbdraw{0x3300}
\caption*{Dhalsim reconstructed sheet. Tiles are 16x16}
\end{figure}

Comparing the released material with what actually shipped is the source of many discoveries and hypothesis. 


The Dhalsim sheet sits at offset \icode{0x3300} in the GFXROM. It is a near perfect match with the paper version except for the portion starting at \icode{0x60}. One of the pose was dropped in favor of Chun-li animation "Hundred Rending Legs" which would indicate it was a later addition. 

 \begin{figure}[H]
\img{rom_sheet_ryu.jpg}
\caption*{Ryu released paper sheet}
\end{figure}

Ryu's sheet \icode{0x4500} allows to guess even more about the production process. Large coherent sprites show that at the beginning of the production process multiple sheet were allocated on a per-character basis. Tile were layout and kept together as much as possible to facilitate visual inspection.

However, as the project progressed, we can see that the team scrambled for tile. They scrapped the bottom of the barrel and started to allocate space on a per-tile basic, sometimes spreading a character pose across multiple sheets like in Dhalsim sheet where can be found portions of Blanka.

 \begin{figure}[H]
\nbdraw{0x4500}
\caption*{Ryu reconstructed sheet. Tiles are 16x16}
\end{figure}


And this was only the allocation step! After a shape was allocated, the artists has to draw it on non-rectangular grilled paper (in order to compensate for the CPS-1 non-square pixels). The last step for the artist was to convert its collection of tiles into graphic files. 

With no scanners available, digitalization was done manually by looking at the paper grid and decide what color would be the most fitting.




The tool used, TCE (Tiny Character Editor), ran on a SMC-70. Sony's machine running on a 4.028 MHz Z-80 with 65KiB had high resolution of 320x200 in 16 colors and no mouse.

\begin{q}{akiman\cite{ar20160404}}
  As everything was in hexadecimal we used the 0-F keys and the arrows to make the sprites. There was this one guy who made a complete racket mashing away on his keyboard. He used to do overtime and didn't even sleep, so we'd all have no choice but to stay awake and keep working as well.
\end{q}

 \begin{figure}[H]
\img{smc-70_capcom.png}
\caption*{An actual SMC-70 "workstation" in Capcom studio}
\end{figure}

\subsection{Team structure and Work ethic}

In its early years, Capcom organized its pipeline to produce multiple games in parallel. Three teams, named first, 2nd and 3rd Planning Rooms worked side by side. 

As it grew and diversified, the gaming company restructured itself in 1988 to be made of two divisions. One dedicated to arcades and the other home consoles. 

After its restructuration, Capcom kept the same modus operantis where each team were siloed from each others with an internal organization favoring a strong sense of hierarchy. 


\begin{q}{akiman\cite{<ref src="http://shmuplations.com/akirayasuda/">Akiman, 2003 Interview from Capcom Design Works</ref>}}At the top are the Planners. Junior artists work on backgrounds, while more senior get to work on sprites.
\end{q}

\begin{q}{Akiman, 2003 Interview from Capcom Design Works}
  We had vacation days, but Okamoto would get mad if you took the day off. A lot of people got yelled at by him for that, “Hey, why weren’t you here on Sunday?!”

  I don’t think anyone can beat my record for “percentage of time lived at Capcom.” During game developments, I always slept under my desk. I had a whole futon laid out and everything! When things were really busy, Yoshiki Okamoto (former Capcom development leader) would be setting new deadlines every 10 hours, so I couldn’t leave my computer… that’s how I acquired the habit of sleeping under my desk. By the way, even now that I’m freelance, I still sleep under my computer desk at home.
  \end{q}


As layered as they where, operations where not set in stone and employees could climb the ladder quickly. 

A key figure in the production of "Street Fighter 2", Akira - Akiman - Yasuda, was hired on "Dyn Side Arms" as a SCROLL artist. Two years later, He was a Planner on Forgotten Worlds.

\begin{trivia}
The role of each person is clearly established. Try to checkout credits on \url{mobygames.com}. You will find OBJects designers and SCRoll designers there.
\end{trivia}


\subsection{Inspiration}


% \begin{figure}[H]
% \img{sf_art_research.jpg}
% \caption*{Original characters}
% \end{figure}

\begin{q}{Akiman\cite{ffdevinterview}}
I remember stitching together a few movies to make a presentation. “Streets of Fire” and Charles Bronson’s “Hard Times” were the ones I used back then. Basically movies about fighting. I really took the chairman’s words to heart – “Use movies!” he said, so I took that to mean we should just openly plagiarize them (laughs).
\end{q}

\begin{q}{Nishitani\cite{ffdevinterview}}
We didn't have a whole lot of time, so we had a 3-monitor set-up where we could watch other movies at the same time, as the president told us to "watch them all and learn from them."
\end{q}

Coincidentally, the Japanese title of "Hard Times" was "The Street Fighter".

 \begin{figure}[H]
\img{ken_stage_design.jpg}
\caption*{Ken stage design. Drawing from Street Fighter II Complete File}
\end{figure}

\section{Shapes and Sprites}
Since we are now accustomed to the concept of sheet, we can review the last GFX ROM requirements.  


\simg{0.8}{honda.png}

 \begin{figure}[H]
\nbdraw{honda_sheet}
\caption*{Honda sheet}
\end{figure}