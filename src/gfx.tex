\chapter{GFX System}
Since the graphic pipeline of the CPS-1 is hard-coded in the silicon of the CPS-A and CPS-B, there is no code to write and nothing to compile. 

\begin{figure}[H]
\sdraw{1.0}{gfx_arch}
\caption*{The GFX System components}
\end{figure}

It sound simple but there is more at hand than converting graphic assets to GFXROM format. Even though all types of assets use the same pixel format, observant readers will have noticed that palettes are not stored in GFXROM. Special care must be taken to save them and provide them later to the m68k along with a way to reference them.

The other constraint to consider is the "hard-coded regions" of the GFXROM where OBJ, STAR1, STAR2, SCR1, SCR2, and SCR3 assets must reside. If a tile is not where it should, a draw command is simply ignored.

All board use a different layout. The Street Fighter II board and its \icode{STF29} PAL slice the 6MiB GFXROM in four areas (OBJ, SCR1, SCR2, and SCR3) . 

A board running Final Fight uses a \icode{S224B} PAL which also carves the space in the same four area types but with different offset and proportions over a smaller 2MiB space. 

The PAL found on Forgotten Worlds board, the \icode{LW621}, is more complex since it divide the GFXROM in five areas to also include interleaved STAR1 and STAR2 bytecode.


\section{Tile format}
All tiles are stored continuously in memory, using groups of four bytes encoding eight pens. Rows of pens are stored one after another. The dimension of a tile varies depending on the layers. OBJ and SCR2 uses the same 16x16 tiles which mean their tiles are 16*16 / 2 = 128 bytes. For the fine tuned SCR1 which uses 8x8, each tile uses 32 bytes. Finally the larger SCR3 uses 512 bytes for each of its 32x32 tiles.

\section{GFX Layout}
The EPROMs are organized in groups of four chips and serialized. On a board like Street Fighter II, we find three groups of four. Inside each group, chips are interleaved every two bytes.

\nbdraw{gfx_interleave}

\section{Channels}\label{channels}\index{Channels!Practice}
The channels mentions earlier are now visible. ROM \icode{01} and ROM \icode{02} are paired on channel 1 to provide 32-bit data. Likewise, ROM \icode{03} and \icode{04} are paired on channel 2. The same division is applied to the rest of the ROMs with the same principle. 

\nbdraw{build_graph_gfx}

Street Fighter II board build generates twelve roms. Each group stores 2 MiB. Group [1-4] start at 0x000000, group [10-13] at 0x200000, and ROMs [20-23] at 0x400000. Each chip goes in the matching numbered DIP slot (page \pageref{boardb_no_chips}).










\section{Back in the days}\index{Back in the days!GFX}

The artwork was the part of a game that required the most people. On a big title such as Street Fighter II, that task kept busy twenty artists out of a team of forty.

Producing tilemap (background) assets for the SCROLL layers was junior artists responsibility. It required little supervision since the artwork had to be rectangular and the visual importance was not the highest.

Producing assets for the OBJ layer on the other hand was much more elaborated. Four steps, outline drawing, allocation, detail led drawing, and finally dotting were involved.



\subsection{Pen and Papers}
The artistic direction was established by the planner of the game. Their goal was to produce concept arts and pose outlines in order to captures the essence of a character as well as establishing proportions and movements. After that, senior artists took over for fine tuning.

\begin{q}{Eri Nakamura\cite{sf2devinterview}}
When I joined on Street Fighter II, Akiman had already done the rough drafts. 

There were four of us as character leads - Satoru Yamashita , Yoshiaki Ohji, and Ikuo Nakayama in addition to myself. Satoru was the most skilled, so he would take Ryu and Ken. 

One day, Akiman brought us the rough sketches of a pro wrestler, sumo wrestler, and a beast, and said "decide who does what." To be fair, we played paper-rock-scissors to determine!
\end{q}



Even though it was no the norm, it could happen that the planner took care of every aspect of a character especially if the topic was dear to their heart.

\begin{q}{Akiman\cite{gameMaestro4}}
I created all Chun-Li’s graphics in just 1 month.
\end{q}

However, "chun-li" stories seems to have been rare occurrences.   In the case of Final Fight, Satoru Yamashita animated Guy and Haggar but Akiman drew the key animations for both of them.



\begin{figure}[H]
\draw{ff_design}
\caption*{Final Fight Guy sprite outlines by Akiman}
\end{figure}




\subsection{Non-square grid paper}
To draw both outline and detailed version of the sprites, artists used a special paper with a double grid system.

There was a "light" grid which used non-square proportions to match the CP-system video aspect ratio of 10:7. Artists were able to draw normally without having to worry about distortion since rectangle elements would match the CPS-1 stretched pixels. 

The paper featured a second "darker" grid which grouped elements 16 by 16 to match the OBJ tile dimension. This was an essential feature for the allocation step.





\subsection{OBJ allocation}

If breaking free of the rectangular sprites was a blessing for the artists, it was a problem for Capcom project managers. In an era where ROM chips were very expensive, a game was allocated a ROM budget at its beginning which it could not exceed.


Before the CPS-1, remaining within the budget was a simple matter of a division. The number of sprites allowed to the art team was ROM size / rectangular sprite size. But the free form factor introduced a tracking problem.



 \begin{figure}[H]
\nbdraw{0x3300}
\caption*{Dhalsim reconstructed sheet}
\end{figure}

The solution to the allocation problem came with papers and scissors.

\begin{q}{Akira Nishitani, Planner \cite{sf2devinterview}}
In order to make the best use of the capacity we had, we wrote the ROM’s capacity on a board, and cut and paste the pixel characters on the board. If there was space left on the board, then there was open capacity in the ROM. So, from there we started filling in the spaces, like a puzzle. 

We saved making the ending for last, and by the time we got there we were all out of capacity. We were wondering what to do, when we found a board that had gone missing under a desk.

We called it the "Mirac-ulous Memory."
\end{q}



 \begin{figure}[H]
\img{rom_sheet_dhalsim.jpg}
\caption*{Dhalsim released paper sheet}
\end{figure}

Only two of these sheets have ever been released, one mostly featuring Dhalsim\cite{ffdevinterview} and another one called the "Ryu sheet"\cite{htmcc}. Thanks to the imprint left in the GFXROM, all other spreadsheets can be reconstructed thank to our knowledge of the pixel format and layout. 



For a game like Street Fighter II, a budget of 6MiB GFX was approved. With 4.6 MiB dedicated to sprites, 144 OBJ sheets were printed. That was a lot at the time and only warranted because the team had managed to score a huge hit with Final Fight on a tiny 2MiB budget\cite{gameMaestro4}. 




 \begin{figure}[H]
\nbdraw{0x4500}
\caption*{Ryu reconstructed sheet}
\end{figure}


Comparing the released material with what actually shipped is the source of many discoveries and hypothesis. 

The Dhalsim sheet sits at offset \icode{0x3300} in the GFXROM. It is a near perfect match with the paper version except for the portion starting at \icode{0x60}. One of the pose was dropped in favor of Chun-li animation "Hundred Rending Legs" which would indicate it was a later addition. 

 \begin{figure}[H]
\img{rom_sheet_ryu.jpg}
\caption*{Ryu released paper sheet}
\end{figure}

Ryu's sheet \icode{0x4500} allows to guess even more about the production process. Large coherent sprites show that at the beginning of the production process multiple sheet were allocated on a per-character basis. Tile were layed out and kept together as much as possible to facilitate visual inspection.

As the project progressed the team scrapped the bottom of the barrel and started to allocate space on a per-tile basic, sometimes spreading a character pose across multiple sheets like in Dhalsim sheet where can be found portions of Blanka.

\subsection{The sheet system}
Besides sparsely describing it, Capcom employees never elaborated on the sheet system. For which title and for how long it was used in total are questions that were never explained.

Thanks to the understanding of the GFXROM format, it is possible to peek back in time. The digital structure is an imprint of what the paper sheet looked like. These reconstructed sheet can provide a few answers.

 \begin{figure}[H]
\nbdraw{sfce0x0000}
\caption*{A Street Fighter II Champion Edition sheet}
\end{figure}

Examining the GFXROM structure of all games published on CPS-1 uniformly reveals tiles grouped to match actual drawing, which imply usage of paper and scissors.

The GFXROM layout started to change with the first title using the CPS-2. In Super Street Fighter II, the sheets of the twelve legacy contestants are the same as the one used in Street Fighter II Champion Edition. However new character sheets look like they were created with an automated allocator.

Likely a transition operated during the production of SSF2 and all games released after September 1993 used automated tools to allocated the OBJ layer tiles.

 \begin{figure}[H]
\nbdraw{ssf-0x9000}
\caption*{Super Street Fighter II sheet (new character Cammy)}
\end{figure}

\img{smc-70_ad.jpg}


\subsection{Digitizing art}\index{Computers!SONY SMC-70}

To digitize their drawings, Capcom employees used SMC-70 computers. Manufactured by Sony, the SMC-70 hit both the US and japan market in the end of 1982. What is particularly noteworthy about this machine is that is built around extensibility. 

The main element of a SMC-70 features its keyboard and the core parts such as the z80 running at 4MHz, 64KiB RAM, and 64 KiB of VRAM. The rest is entirely configurable via daisy chained extenders. 

The only limit to the daisy chain extension system is the capacity of the power supply which also must be located at the very back of the chain. 


This architecture allowed Sony's machine capabilities to range from simple office work to a powerful video editing tool in its most extensive modification. 

\begin{figure}[H]
\img{sm70_drawing1.png}
\caption*{A SMC-70 extended with a SMI-7012. Power supply in the back}
\end{figure}

\subsubsection{Sony, the big iron}
On one side, the substantial list of extensions and peripherals, totaling nearly 40 pieces made by Sony, is a testimony to the company's commitment. One the other side, it also embodies a desire to be the "big iron" where everything is under control, certified, and others manufacturers are prevented from contributing.

\begin{figure}[H]

\begin{tabularx}{\textwidth}{rrX} 
  \toprule 
  \textbf{Extension Name} & \textbf{Peripheral Name} & \textbf{Function} \\               
  \toprule    
\texttt{SMI-7011} & & 3.5" floppy drive bay (internal with 1 drive)\\ 
\texttt{SMI-7012} & & 3.5" floppy drive bay (internal with 2 drives)\\ 
\texttt{SMI-7013} & & 3.5" floppy drive bay (external with 1 drive)\\ 
\texttt{SMI-7014} & & 3.5" floppy drive bay (external with 2 drives)\\ 
\texttt{SMI-7016} & & Floppy Disk Control Unit\\ 
 & \texttt{SMI-7020} & Dot Matrix Printer\\ 
\texttt{SMI-7031} & & RS232C Serial Interface\\ 
\texttt{SMI-7032} & & IEEE-488 Interface Unit\\ 
\texttt{SMI-7050} & & Cache Disk Unit\\ 
\texttt{SMI-7056} & & Supercharger: 5MHz i8086 w/ 256 KiB RAM.\\ 
 & \texttt{SMI-7060} & 10-Key Numeric Key Pad\\ 
\texttt{SMI-7070} & & Video Signal Converter\\ 
\texttt{SMI-7073} & & RGB Superimposer\\ 
\texttt{SMI-7074} & & NTSC Superimposer\\ 
\texttt{SMI-7075} & & Videotizer\\ 
\texttt{SMI-7080} & & Battery Back-up Unit\\ 
\toprule
\end{tabularx}%
\caption*{SMC-70 extensions and peripherals\cite{smc70tech}}
\end{figure}


\subsubsection{Noteworthy capabilities}
The SMC-70 is notable for being the first computer to allow 3.5" floppy reader (also invented by Sony in 1981) and its ability to display kanji character via a ROM extension. 

However it is really when it comes to graphic capabilities that the machine stood out. Four resolutions were available, ranging from low-res 320x200 using sixteen colors up to high-res 640x200 in two colors. 

The 16 colors color mode was particularly interesting since it was a perfect match with the CPS-1 pen system.


\subsection{Tiny Character Editor}

The SMC-70 had no ability to use a scanner. The digitization process was entirely achieved by hand. To help them in their task, artists used a tool called TCE (Tiny Character Editor). Although no screenshot ever merged, Capcom employees gave a rough description of the minimalist approach of their tool.

\begin{q}{Koichi Yotsui (Strider planner)}
You had a 16 pixel grid, a 16-color palette, and that was it.
\end{q}


\subsection{Dotting}
For both SCROLLs and OBJs element, dotting was done on a tile basis. The artist task was to look at their detailed grilled paper drawing and decide, for each rectangular element in the tile, which color of the palette to use for it.

It was a tedious process which paradoxically also required a sense of esthetic and creativity to deal with the cases where a line crossed a pixel. Drop the pixel, include it plain, or attempt to anti-alias with a color in the palette were difficult choices to make.

% The task at hand was to look at the detailed drawing, and decide which color from the palette would be used for this coordinate. The value was then set via TCE. 

No mouse were available. Out of the box options were either a keyboard or a keypad. Some employees, satisfied with neither built themselves a joystick.


\begin{figure}[H]
\img{smc70_keypad.jpg}
\caption*{A SMC-70 with a keypad. A likely doter setup}
\end{figure}


At least, employees were free to use the technique that was the most effective to them.

\begin{q}{Akiman\cite{akiman}}
I used a keyboard to draw all the graphics for Vampire and Street Fighter 2.
\end{q}



\begin{q}{Akiman\cite{ar20160404}}
  As everything was in hexadecimal we used the 0-F keys and the arrows to make the sprites. 

  There was this one guy who made a complete racket mashing away on his keyboard. He used to do overtime and didn't even sleep, so we'd all have no choice but to stay awake and keep working as well.
\end{q}


\subsection{Saving tiles}
Artists attempted to reuse tiles as much as possible to reduce the precious ROM space available. In Street Fighter 2, there was only enough GFXROM for eleven challengers. Ken is a patchwork and a palette swap on top of Ryu tiles base. It "weights" only 98,304 bytes. A remarkable achievement compared to characters such as Zangief (622,592 bytes), Honda (491,520 bytes) or Ryu (442, 368 bytes).

\begin{minipage}[t]{0.19\linewidth}
  \sdraw{1.0}{patch_ryu0}
\end{minipage}%
\hfill%
\begin{minipage}[t]{0.19\linewidth}
  \sdraw{1.0}{patch_ryu1}
\end{minipage}
\hfill%
\begin{minipage}[t]{0.19\linewidth}
  \sdraw{1.0}{patch_ryu2}
\end{minipage}%
\hfill%
\begin{minipage}[t]{0.19\linewidth}
  \sdraw{1.0}{patch_ryu3}
\end{minipage}
\hfill%
\begin{minipage}[t]{0.19\linewidth}
  \sdraw{1.0}{patch_ryu4}
\end{minipage}


Made with 2MiB GFXROM, Final Fight is more impressive with a 21 enemies and 6 bosses. The minions are made of seven bases, diversified with patches and palettes.

\begin{figure}[H]
\nbdraw{ff_template}
\caption*{G. Oriber, Bill Bull, and Wong Who from Final Fight}
\end{figure}


\begin{q}{Nishitani, Capcom – 1991 Retrospective Interview}
Everyone on the development thought Final Fight was going to be allocated with a large memory capacity, but we were wrong.
That's why the final boss Belger hops around like that: we didn't have enough memory to add more graphics for a walking pattern. 

However, making something cool with limited resources is like a puzzle to me, so I thought it was fun. 
\end{q}


 \begin{figure}[H]
\nbdraw{0x0100}  
\caption*{Ryu/Ken sheet}
\end{figure}

Ryu and Ken use the same seven first colors of their palette to facilitate the patching process. 

\nbdraw{palette_ryu}


\nbdraw{palette_ken}



 \begin{figure}[H]
\nbdraw{0x4e00}
\caption*{Sagat's sheet}
\end{figure}

Sagat's laughing animation is double optimized. The sequence is made of two poses where only the bust is replaced while the legs remain the same. Moreover, the left leg is missing. It is reconstructed at runtime using an horizontal mirror of the right leg found at \icode{0xB9}.

\begin{trivia}
The capacity of the ASICs to flip tiles horizontally was used extensively in Street Fighter II when challengers facing left or right. Not really an issue for symmetrical characters, except for Sagat who' eye patch changes side when he turns.
\end{trivia}

\pagebreak

\subsection{Team structure and Culture}

The team producing a game had a strong hierarchy based on skills and seniority.

\begin{q}{Akiman\cite{akiman2003}}Planners at the top.\\
Senior artists get to work on sprites. \\
Junior artists work on backgrounds.
\end{q}

The structure was flat with no involvement of intermediate managers. Out of the twenty people doing artwork on Street Fighter II, all of them reported to a single person, Akiman\cite{sf2_oral_history}.




As layered as they where, operations where not set in stone and employees could climb the ladder quickly. Akiman was hired on "Dyn Side Arms" as a SCROLL artist in 1986, at the bottom of the artist ladder. Two years later, he was a Planner on Forgotten Worlds and went on to work on Final Fight and Street Fighter 2 in the same capacity.

\subsubsection{Work Ethic}

A strong working culture was established from the very top.

\begin{q}{Akiman\cite{akiman2003}}
  We had vacation days, but Okamoto (Capcom development leader) would get mad if you took the day off. A lot of people got yelled at by him for that, "Hey, why weren't you here on Sunday?!"

  I don't think anyone can beat my record for “percentage of time lived at Capcom.” During game developments, I always slept under my desk. 

  I had a whole futon laid out and everything! When things were really busy, Yoshiki Okamoto  would be setting new deadlines every 10 hours, so I couldn't leave my computer… that's how I acquired the habit of sleeping under my desk. 

  By the way, even now that I'm freelance, I still sleep under my computer desk at home.
  \end{q}

\subsubsection{Poaching}


Retaining talent was a top priority. The credit screen of Street Fighter II illustrates Capcom cautiousness. Artists were only credited by their nicknames.

 \begin{figure}[H]
\img{sf2-credit.png}
\caption*{Street Fighter II credit screen does not use actual names for fear of poaching.}
\end{figure}

\begin{trivia}
The most recognition available was a specialty next to a nickname. Some "credit screens" feature \textbf{OBJ}ects artists and \textbf{SCR}oll artists.
\end{trivia}

% \begin{trivia}

% \end{trivia}





Despite its precautions, Capcom lost numerous employees over the years. Among them was Takashi Nishiyama who made Street Fighter 1 and then went on to direct Fatal Fury' The King Of Fighters for Capcom arch-rival, SNK\cite{YoshikiOkamotoTakashiNishiyama},.
 

% In its early years, Capcom organized its pipeline to produce multiple games in parallel. Three teams, named first, 2nd and 3rd Planning Rooms worked side by side. 

% As it grew and diversified, the gaming company restructured itself in 1988 to be made of two divisions. One dedicated to arcades and the other home consoles. 

% After its restructuration, Capcom kept the same modus operantis where each team were siloed from each others with an internal organization favoring a strong sense of hierarchy. 











\subsection{Inspiration}
For Street Fighter II, artists inspiration came from various outlets. 

Mangas such as "Yasunori Katō" helped to give birth to Dictator while Tao from "Harmagedon: Genma Wars" was part of the genesis of Chun-Li. 

Boxer. Ryu, Ken, Sagat, and Zangief were inspired by real life athletes, respectively Mike Tyson, Mas Oyama, Joe Lewis, Sagat Petchyindee, and Victor Zangiev Zhanghief.

\begin{trivia}
Originally called M. Tyson, the boxer was renamed to "Balrog" for the US release, out of lawsuit concerns. 
\end{trivia}

% \begin{figure}[H]
% \img{sf_art_research.jpg}
% \caption*{Original characters}
% \end{figure}

For the backgrounds, Hollywood and VHS cassettes came to the rescue.

\begin{q}{Akiman\cite{ffdevinterview}}
I remember stitching together a few movies to make a presentation. "Streets of Fire" and Charles Bronson’s "Hard Times" were the ones I used back then. Basically movies about fighting. 

I really took the chairman’s words to heart – "Use movies!" he said, so I took that to mean we should just openly plagiarize them!
\end{q}

Employees did not get paid to a watch movie. They got paid to watch three movies at the same time!

\begin{q}{Nishitani\cite{ffdevinterview}}
We didn't have a whole lot of time, so we had a 3-monitor set-up where we could watch other movies at the same time.

We did as the president told us: - "Watch them all and learn from them!".
\end{q}

\begin{trivia}
Coincidentally, the Japanese title of "Hard Times" was "The Street Fighter".
\end{trivia}


\section{Shapes and Sprites}
This historical detour was important in order to understand the sheet system. With this knowledge we can review the last GFX ROM requirements involving OBJs.  

On this layer, tiles can be used either directly or in groups which mandate distinct layout in GFXROM.

\begin{figure}[H]
\img{ken_stage_design1.png}
\caption*{Ken stage inspiration.}
\end{figure}

\begin{figure}[H]
\img{ken_stage_design2.png}
\caption*{Ken stage sketch\cite{sf2completefiles}.}
\end{figure}

\begin{figure}[H]
\img{ken_stage_design3.png}
\caption*{Ken stage as seen in the game.}
\end{figure}


\subsection{Sprite}
A sprite is a collection of tiles with rectangular boundaries. As we will see in the m68k programming section it can be rendered by issuing a single draw call mentioning the offset in the sheet, the width in tiles and the height in tiles.

 \begin{figure}[H]
\nbdraw{honda_alloc1}
\caption*{Honda has a Sprite}
\end{figure}

While it is convenient to be able to use a single command and the best way to render a set of mostly opaque tiles, it is inefficient. A set of tiles where many of them are transparent not only waste precious storage space in the GFXROM, they also count against the CPS-A/CPS-B limit of 256 tiles per frame.

Another limitation (or advantage depending how you look at it) is that a single palette is specified when placing the sprite draw command.

\subsection{Shape}

A much more efficient and flexible way is to use a Shape where tiles layout can is allowed to be arbitrary. It takes several draw calls to draw (tiles have to be specified one by one) but they can be located anywhere in a GFXROM.

Shapes have the triple advantage of saving storage space, narrow down tile count during rendition to the minimum, and allow per tile palette. The example of Honda shows that only 41 tiles are required. It would have taken 60 tiles if a Sprite had been used.

\begin{trivia}
In Street Fighter II artists limited themselves to one palette per character because the OBJ layer was used for multiple things such as decoration and GUI. This was purely a design decision, they could have as well gone with 16 palettes per characters since they are drawn with shapes instead of sprites commands. 
\end{trivia}

 \begin{figure}[H]
\nbdraw{honda_alloc2}
\caption*{Honda as a Shape}
\end{figure}


To allow sprite draw calls, the build system must allocate images depending on the intended usage. All tiles in a sprite must be placed as they appeared in the original image.
 

Capcom games use almost exclusively shapes but the paper sheet system mandated to maintain some visual coherency so humans could keep tracks. Modern tools don't have this problem. 

On sheet page \pageref{honda_sheet} is a Honda sheet featuring the same artwork twice. It is features first as a  Sprite, at \icode{0x00} where it appears like a rectangle. It appears a second time as a Shape. Thanks to the automatic allocator in \icode{ccps}, tiles are placed as they were encountered when reading the asset. It looks like mashed potatoes but use less space.

\begin{figure}[H]
\nbdraw{honda_sheet}
\caption*{A sheet, generated with \icode{ccps}, with a Sprite Honda and a Shape Honda}
\end{figure}
\label{honda_sheet}

% On the contrary, a shape does not have to follow any constraint.




