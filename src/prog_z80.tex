\chapter{Sound System}
Making ROMs for the sound system is a little bit more complicated than for the GFX system. 

Not only are samples and music assets processed, there is also a need to compile code to bootstrap and then run the z80. This program is commonly called a "Sound Driver".

\begin{figure}[H]
\sdraw{1.0}{sound_arch}
\caption*{The Sound System components}
\end{figure}

The assets and the driver must be packed into two ROMs, one called "z80 ROM" and another called "OKI ROM". This asymmetry where three systems result in two ROMs is further complicated by the dependency graph that looks like a plate of spaghetti. 

The difficult part is that both sound effects and music contribute to the OKI ROM content which must be populated in two steps.



\nbdraw{build_graph_sfx}

The structure of the OKI ROM facilitates the task since it features an index at the beginning which references all payloads. Leveraging it allows a first pass where audio samples are ingested by the build system to generate an incomplete OKI ROM. In a second pass, the music tracks are processed so new samples are added to the index ROM and the payload appended.

At the very top, \icode{ccps\_sfx} expects artists to provide sound samples contained in \icode{.wav} files, a format which is universally supported by audio tools. This stage generates a partial OKI ROM and a \icode{.h} header file to be injected in the control build graph so the m68k can request sample playback with a simple ID.

The second stage involves \icode{ccps\_mus}, the music processor. It expects \icode{.vgm} (Video Game Music) from which are extracted raw YM2151 instructions. It outputs the final OKI ROM with the sample used by the soundtrack and also a \icode{.c} file including bytecode containing YM2151 register values, timing, and OKI sample timing.

The last stage relies on the Small Device C Compiler (SDCC) toolchain. All artifacts generated in the previous stages are used along with the inputs for the sound driver (\icode{.c} files) and the bootstrap \icode{crt0.s} assembled by \icode{sdasz80} assembler. In this step, all resulting relocatable object files \icode{.rel} are linked together via ssdc's linker \icode{sdldz80}. 





\section{Processing Sound Samples}
A wav file is a simple container with a header describing the content followed by a payload. Once the sampling frequency, bits per sample, and number of channels is retrieved, the PCM can be accessed.

\subsection{Constraint}
Artists should not produce stereo wavs since the CPS-1 is mono. Moreover, game developers should decide if they wish to use the OKI in high quality (7575Hz) or low quality (6060Hz) and all assets should use that sampling rate. Finally, since ADPCM compressed 12-bit sample to 4-bit samples, artists should provide 16-bit wavs.

\subsection{ADPCM Compression}
\index{ADPCM!Compression}
Decompression is done in hardware by the OKI at runtime but the build system still has to compress assets appropriately. 

ADPCM encodes the difference between the previous sample and the sample to be generated. Each 16-bit sample is downsampled to 12-bit and compressed to a 4-bit nibble\cite{adpcm_specs}.


\lstinputlisting[style=CStyle]{src/code/adpcm_algo.c}


Let's take an example converting a stream of 16-bit PCM to 4-bit ADPCM nibbles.

\lstinputlisting[style=CStyle]{src/code/pcm-dpcm.c}

\subsubsection{Compressing sample 1}
The first 16-bit sample has a value of \icode{960} which becomes \icode{60} in 12-bit. With the current step size at 16, ADPCM can only command a delta of + ( 16 + 16/2 + 16/4) = +28 which it encodes in a nibble \icode{0b0111}. The step size index is updated via the transitionTable[\icode{b111}] = 8. The current step size is \icode{34}. 

\subsubsection{Compressing sample 2}
The second 16-bit sample also has a value of \icode{960} which becomes \icode{60} in 12-bit. Since the latest sample output was 28, ADPCM must somehow encode a difference of 60-28 = 32. ADPCM commands a delta of + ( 0 + 34/2 + 34/4) = +25 which it encodes in a nibble \icode{0b0011}. 

The decompressor will output 28 (its last value) + 25 = 53. We can see how the step size has adapted to the delta requested with only two steps. 

The step size index is updated via 8 (previous value) -1 (transitionTable[\icode{b011}]) = 7. Now the step size is \icode{31}.

\subsubsection{Compressing sample 3}
The third 16-bit sample has a value of \icode{950} which becomes \icode{59} in 12-bit. Since the latest sample output was 53, ADPCM encodes a difference of 59-53 = 6. ADPCM commands a delta of + (0 + 0 + 0) = 0 which it encodes in a nibble \icode{0b0000}. 

The decompressor will output 53 (its last value) + 0 = 53. The step size index is updated via 7 (previous value) - 1  (transitionTable[\icode{b000}]) = 6. Now the step size is \icode{28}.

\subsubsection{Compressing sample 4}
The last 16-bit sample has a value of \icode{160} which becomes \icode{10} in 12-bit. Since the latest sample output was 53, ADPCM encodes a difference of 10-53 = -43. ADPCM commands a delta of - (28 + 28/2 + 0) = -42 which it encodes in a nibble \icode{0b1110}. 

The decompressor will output 53 (its last value) - 42 = 11. The step size index is updated via 6 (previous value) + 6 (transitionTable[\icode{0b110}]) = 12. Now the step size is \icode{50}.

Graphing PCM vs ADPCM shows that the decompressed stream "lags" behind upon abrupt changes but catches up aggressively. ADPCM first value is always near zero (+/-28) but it is not audible to players since most samples start with a fade-in. 

\nbdraw{adpcm_graphed}

\section{Structure of the OKI ROM}
The structure of the OKI is simple. It features a 127 * 6 bytes table at its beginning. Each entry points to a payload in the ROM. The payload of each entry must be a ADPCM stream.

For some reason offset 0 in the table index must not be used.

\lstinputlisting[style=CStyle]{src/code/okistruct.c}

Notice there is no metadata to indicate the bitrate. The "database" entry is a plain offset directly pointing to ADPCM nibbles. It is up to the build system to carry out this info to the z80. In practice, Capcom games never mixed bitrate and always used 7575Hz.

\begin{figure}[H]
\nbdraw{oki_rom}
\caption*{OKI ROM layout\cite{MSM6295_datasheet}}
\end{figure}















\section{Processing Music}
Making music with FM synthesis is an art which this book does not intend to butcher. The best way to go is to ask the musicians in the team to work with whatever tool they wish (the amazing DefleMask is highly recommended) and export their symphonies using VGM file format.

VGM (Video Game Music) is a community effort originating from \icode{smspower.org} to create an audio file format able to support many legacy systems (SEGA consoles, MSX, Neo Geo, and PC) as well as arcade hardware.

The arcade profile is particularly interesting for CPS-1 development since it uses YM2151 for FM Synthesis and SegaPCM for the samples.

While the YM2151 is obviously a perfect fit that will need no processing, the SegaPCM needs some adjustments. As a chip used by SEGA in their AM2 (Amusement Machine 2) from 1985 to 1991, it is superior in capabilities to the MSM6295. It relies on 16-bit PCM, up to 32kHz sampling, has more channels, and has a larger address space.

The build system can take care of compressing the SegaPCM to ADPCM and resampling from 32KHz to 8080Hz but the musicians should only use two channels for music samples in order to play nicely with the CPS-1, where the two other channels are used to play sound effects.


\lstinputlisting[]{src/code/z80/vgm.c}



\section{Programming the z80}
Programming Zilog's CPU into a sound driver involves doing two things right: bootstrapping and setup of the memory map. Since the latter is easier, it is discussed first. To avoid page turning, the requirements are reproduced here.

\begin{figure}[H]
{
\begin{tabularx}{\textwidth}{rrrX}
\toprule    
  \textbf{Start } & \textbf{End  } & \textbf{Size } & \textbf{Function } \\               
  \toprule    
  \texttt{0x0000} & \texttt{0x7FFF} & 32 KiB & ROM (32 KiB out of 64 KiB)\\
  \texttt{0x8000} & \texttt{0xBFFF} & 16 KiB & Bank-switched view of rest of ROM\\
  \toprule    
  \texttt{0xD000} & \texttt{0xD7FF} & 2 KiB & RAM \\
\toprule    
  \texttt{0xF000} & \texttt{0xF001} & 2 B & YM2151 registers\\
  \texttt{0xF002} & \texttt{0xF002} & 1 B & OKI OKI6295 registers\\
  \texttt{0xF004} & \texttt{0xF004} & 1 B & Bank Switch control (\icode{SOU1})\\
  \texttt{0xF006} & \texttt{0xF006} & 1 B & OKI MSM6295 H / L mode\\
  \toprule    
  \texttt{0xF008} & \texttt{0xF008} & 1 B & Sound commands (latch 1)\\
  \texttt{0xF00A} & \texttt{0xF00A} & 1 B& Sound commands (latch 2)\\
  \toprule    
\end{tabularx}%
}\caption*{z80 memory map}
\end{figure}


Using \icode{sdcc} helps a lot here since it features an exclusive "placement" keyword \icode{\_\_at}.

\lstinputlisting[style=CStyle]{src/code/z80/memory_map.c}

With the registers correctly mapped, what remains is to place \icode{.text} and \icode{.data} at the right locations in ROM. This can easily be done thanks to the awesome \icode{sdldz80} linker and its linker script.

\lstinputlisting[]{src/code/z80/main.lk}

As expected, the \icode{\_DATA} is placed at \icode{0xd000}. The \icode{\_CODE} however is placed not at \icode{0x0000} but \icode{0x0200} for reasons that will follow soon.

\begin{trivia}
Debugging a CPS-1 program can be a tedious task. A good starting point when encountering an issue is to read the linker \icode{.map} file which indicates where each symbol was placed.
\end{trivia}

\subsection{Bootstrapping}
A z80 starts fetching and executing instructions from address \icode{0x0000}. The bootstrap code \icode{crt0.s} (on page \pageref{z80_crt0}) is placed accordingly via directive \icode{.org 0}. The code immediately jumps to \icode{0x100} in order to skip the interrupt handler instructions. 

The z80 can work in interrupt modes 0, 1, or 2. Modes 0 and 2 are the most powerful and complex but they imply retrieving the ID of the interrupting peripheral by reading a byte on the data bus. This mechanism allows support of multi-device interruption. However in this case, it is overkill. The z80 uses Mode 1 which always makes the CPU jump to \icode{0x38} when interrupted.

At \icode{.org 0x100}, the stack pointer \icode{sp} is set to point at the end of \icode{RAM} (the z80 stack grows downward), the first interrupt is requested and a mystery \icode{gsinit} function is called. All the code in \icode{crt0.s} accounts for a few hundred bytes. Which explains why we requested the linker script to place \icode{\_CODE} further at \icode{0x200}.

\begin{trivia}
The calling functions from ASM to C require using \icode{\_} prefixed symbols.
\end{trivia}
\pagebreak

\lstinputlisting[style=Z80Style]{src/code/z80/crt0.s}
\label{z80_crt0}







\subsection{z80 interrupt}
\index{Interrupts!Programming z80}
In order to interact with the latches properly but also be able to keep track of wall-time, the z80 needs to be interrupted regularly. Zilog's CPU does features a timer RFSH but it is intended for DRAM refresh (which the sound system does not feature anyway).

Instead the interrupts are triggered by the YM2151, thanks to its two internal timers. Timer A is a 10-bit counter while Timer B is an 8-bit counter. For a YM2151 running at 3,579Hz, the trigger formula is 64 * (1024 - value) / 3579.

Setting the Timer A to \icode{800} will result in an interrupt 64 * (1024 - 800) / 3579 = 4ms later. When the YM2151 counter reaches zero, it asserts a line connected to the z80 \icode{INT} line which makes the CPU jump to address \icode{0x38}.

\lstinputlisting[style=CStyle]{src/code/z80/interrupts.c}


\subsection{Initializing variables}
To finish bootstrapping, \icode{crt0} makes sure initialized C variable values are set. The linker placed all values requiring initialization in a \icode{\_GSINIT} segment. By wrapping it with markers \icode{\_INITIALIZER} (src) and \icode{\_INITIALIZED} (dst), they can be copied easily.

\lstinputlisting[style=Z80Style]{src/code/z80/initVar.s}

Copying is done with \icode{ldir} instruction using the linker macros. Prefix \icode{s\_} refers to the "start" of a segment while prefix \icode{l\_} refers to the "length" of that segment.

\lstinputlisting[style=Z80Style]{src/code/z80/copyvar.s}
 

% No initialization of BSS


\subsection{z80 Sound Driver}
The sound driver is a simple loop which reads bytecode from our mini-vgm format to feed music notes, no-op on pauses, and forwards sample playback. 


\lstinputlisting[style=CStyle]{src/code/z80/music.c}

A single latch is used to received commands as "immediate value". If bit 0x80 is set, control is requesting a sound effect to be played, otherwise it is music. 

The \icode{main} uses an "active" loop in order to never outpace the \icode{interrupt} function call frequency. This guarantees it runs in the vicinity of 250Hz. 

\lstinputlisting[style=CStyle]{src/code/z80/driver.c}



\section{Back in the days}\index{Back in the days!SFX/Musics}
From its inception, Capcom recognized the crucial importance of sound for a video-game and set out to hire the very best musicians they could find.

By 1989, they had assembled a pool of talent who called themselves the "Capcom Sound Team".

\img{capcom_sound_team.png}

The Capcom Sound team. L-R: Yoko Shimomura, Yoshihiro Sakaguchi, Manami Matsumae, Masaki Izumiya, Yasuaki Fujita, Mari Yamaguchi, Minae Fujii, Toshio Kajino, and Isao Abe. Identity of table-man is unknown.

\subsection{Recruiting}
Capcom actively recruited by taking advantage of 'careers days' to get graduates to come work for them. Many musicians emerged from music schools located in the Kansai region encompassing Kyoto, Osaka, and Kobe near Capcom headquarters. 

Several alumni of Osaka College of Music ended up working for Capcom, where they were able to compose music for a living while having the security of working for a large Japanese company. 





One of these recruits, Yoko Shimomura, would go on to compose music for games such as Street Fighter II, Final Fight, and Final Fantasy. She gave numerous interviews which help to paint a picture of the life of a musician working at Capcom.


\begin{q}{Yoko Shimomura, Capcom Sound team\cite{beep199010}}

I studied piano in college, but I loved the Famicom, and would often stay up all night playing it. 

Then the next day my shoulders would be all stiff, and my piano teacher would scold me, and my Mom even said “I don’t remember raising a daughter like this.”

I decided that when I graduated, I would go work at a place where I could play both music and Famicom all day without complaints!
\end{q}


Yoko's account of her hiring interview confirms that Capcom was more interested in hiring talented musicians than tech savvy people.

\begin{q}{Yoko Shimomura, Capcom Sound team\cite{beep199010}}
  
I did not know you could write music with a computer until I joined the company. At the entrance exam, I was asked “what sequencer do you use?” and I had to ask back "What? Is that like an electronic controller?".

They had to teach me from the ground up, and after that it was less musical practice than it was technical. The first music data I turned in was thoroughly corrected, and I was feeling really glum.

I was asked to talk about what I knew about FM generation at the entrance exam. I had no idea what it was, so I thought about AM/FM radio and wrote down "it sounds better these days than it used to."
\end{q}



\subsection{Creative process}
Even though musicians were part of a "Sound Team", they usually worked alone on a game. 

They could pick projects based on availability\cite{sf2musics} but they were assigned to the next one immediately after they were done with the previous one.


Being dropped in a project and asked to write music was difficult. It was mostly the planner's responsibility to brief the musician on what kind of music they wanted\cite{sf2_music}.

\begin{q}{Yoko Shimomura, Capcom Sound team\cite{sf2_oral_history}}
NiN would come up to me and show me designs of the characters and explain the personalities of the characters and ask me to make theme songs for each character.

I would look at the backgrounds and the character descriptions and all that, and I noticed that each character had a unique background. And because of that, I suggested making each theme song based on their background country and culture.
\end{q}

The work ethic of the composers had nothing to envy to the artists drawing pixels.

\begin{q}{Yoko Shimomura, Capcom Sound team\cite{sf2musicsecurity}}
At 11 o'clock, all the security was activated and you could not move between floors. The elevators stopped moving too. If you had to finish work by 7 o'clock the next day, we had to pass ROM down from the window on a string.

Once I tied a carrier bag and put the ROM inside and gave it to them. 15 mn later, the phone rang. They said "sorry but we broke the pins when we put it in! Can you give us another?"
\end{q}



\subsection{Tools}
Yoshihiro Sakaguchi, author of Mega Man music, explained what computer the musicians connected to their Yamaha keyboards.

\begin{q}{Yoshihiro Sakaguchi\cite{yoko_shimomura_interview}  }
We worked on both the music and sound effects for Capcom’s games. We’ve got a centralized recording system setup on a PC-98, so that even if we’re writing music for different hardware, we can compose without needing to be able to program.
\end{q}

\subsubsection{NEC PC-9800 series}\index{Computers!NEC PC-98}

NEC entered the personal computer market in 1979 with its 8800 series. These machines, built around 8-bit z80 CPUs would later be known as "PC-88". 

Thanks to its optional kanji ROM, NEC quickly gained traction. The PC-88 accounted for 40\% of the Japanese personal market by 1981.

By the mid-80s, the aging series was discontinued in favor of machines based on 16-bit Intel CPUs such as the i286. The NEC's next computer, named 9801, was be the first in the 9800 line. These machines were commonly referred to as "PC-98".

\begin{figure}[H]

\img{PC9800.jpg}
\caption*{PC-9801, the first in the 9800 series (1984)}
\end{figure}

Like the PC-88, the PC-98 enjoyed considerable success. Over its lifespan ranging from 1992 to 2000, NEC sold more than 18 million units. Across multiple lines such as "Desktop", "Hi-end", or "Laptop", NEC released a new machine every year\cite{9800lines}. 

The success was such that the series accounted for 60\% of the Japanese Personal Computer market by 1991. 



\subsubsection{What Capcom used}
Because so many models were released, it is hard to tell for sure which PC-98 was used for a particular game. What can be done is to list the models released each year in order to get a rough idea.

\begin{figure}[H]
\nbdraw{pc98lines}
\caption*{A selection of PC-98s from 1985 to 1992}
\end{figure}

It is likely musicians working on CPS-1 titles were provided with computers from the Main series. Around that time, it would have been a computer based on a Intel 386 CPU with 50 MiB HDD.


\subsubsection{Proprietary technology}

Despite their name, NEC's machine had nothing to do with the IBM PC. Due to its operating system, MS-DOS, lacking support for Japanese glyphs, Big Blue’s machines never managed to break into the
Japanese market. NEC's PCs were named after what they were, \textbf{P}ersonal \textbf{C}omputers.


\subsubsection{C-Bus}
The PC-98 uses a proprietary 16-bit C-bus instead of the IBM's ISA bus. BIOS, I/O port addressing, memory management and graphics output are also different. This architecture was both a moat that protected NEC from clone manufacturers (which plagued IBM in the USA) and a dungeon that prevented its machine from benefiting from the many peripherals built for IBM PCs.


It was an extra effort for manufactures to create C-bus version of their card but companies such as Roland and Creative did release some of their sound cards for PC-98.

\begin{figure}[H]

\nbimg{roland_cbus2x.png}
\caption*{A C-bus Roland MIDI card for PC98II}
\end{figure}



\subsubsection{Video chip}
Besides their proprietary bus, PC-98s were noteworthy for their, at the time, powerful graphic system.

The heart of it was the High-Performance Graphics Display Controller 7220, more commonly known as $\upmu$PD7220. The PC-98 used two of them, both running 2.5MHz. One handled the 8 KiB VRAM for text while the other acted as a co-processor managing a 96 KiB VRAM framebuffer.

The tandem was one of the first GPUs, presenting primitives to draw lines, circles, arcs, and character graphics. In its highest resolution mode the PC-98 reached an impressive 640×400 with 8 colors. It allowed Latin alphabetic, numeric and most importantly katakana characters. An optional ROM board added 3,000 kanji glyphs to the repertory.   

\subsubsection{The end of PC-98}

Although having specs far inferior to the Fujitsu FM Towns and Sharp X68000, NEC enjoyed tremendous success, selling 18 millions units from 1982 to 1999.


A list accounting for games-only still totals 1,228 titles. Proof of the machine's status as a game developer's favorite.

NEC's domination only started to wane when DOS/V, a special version of Microsoft's MS-DOS supporting Japanese characters, came out at the end of 1990.


% \begin{trivia}
% When it came to record voices, the whole Street Fighter II development team was encouraged to contribute.

% \begin{q}{VGDensetsu\cite{sf2samples}  }
% Chun-Li Spinning Bird Kick line was spoken by Eri Nakamura, the graphic designer in charge of Honda, while a background designer did her victory cry ("Kyahahaha! Yattaaa") and a third person recorded another line.

% For Ryu and Ken, who share the same voice, the graphic designer in charge of Guile lent his voice for the Hadoken and Shoryuken while the one in charge of Boxer did the Tatsumaki Senpukyaku.  
% \end{q}
% \end{trivia}