\chapter{Programming Sounds}

Making a CPS-1 generate sound involves programming the Z-80, producing samples and producing music tracks. Because these three parts are packaged into only two ROM sets, they are not completely straight forward to build. 

 \begin{figure}[H]
\sdraw{0.8}{sound_deps}
\caption*{Three parts, two ROMS}
\end{figure}


Since both sound effects and music samples contribute to its content, the OKI ROM must be generated in two steps.

This is considerably facilitated by the structure of the target ROM which features a small database. Leveraging it allows to have a first pass where audio samples are ingested by the build system to generate an incomplete OKI ROM. In a later pass, the music tracks are processed. The new samples are simply added to the database.

\nbdraw{build_graph_sfx}

The build graph involves three phases driven by three tools, three resource types and a myriad of intermediate artifacts. 

At the very top, \icode{ccps\_sfx} expect artists to provide sound samples contained in \icode{.wav} files which are universally supported. This stage generates a partial OKI ROM and a \icode{.h} header file so the Z-80 code can reference samples with a simple ID.

The second stage involves \icode{ccps\_mus}, the music track processor. It expects \icode{.vgm} (Video Game Music) as input since it contains raw 2151 instructions which are easy to convert. It output a finalized OKI ROM containing the sample used by the soundtrack and also a \icode{.c} C file including a custom bytecode containing YM2151 register values, timing, and OKI sample timing.

Finally, the last stage relies on the Small Device C Compiler (SDCC) toolchain. All artifacts generated in the two previous stages are used as input on top of the programmer \icode{.c} files and the bootstrap \icode{crt0.s} assembled by \icode{sdasz80} assembler. All resulting relocatable object files \icode{.rel} are linked together via ssdc's linker \icode{sdldz80}. 


\begin{trivia}
These are a lot of step but if you use \icode{ccps} build system, it is all taken care of with one keystroke resulting in two ROM sets ready to go.
\end{trivia}

\subsection{Z-80 bootstrap (crt0)}

A Z-80 starts in the simplest way by executing instruction at address \icode{0x0000}. Since the ROM will be mapped at \icode{0x0000-0x7FFF} in Z-80 address space there is no offset to factor in when writing the bootstrap. We write a small piece of assembly and request to have it placed at zero (\icode{.org 0x0000}) at features in \icode{crt0.s}.
\pagebreak

\lstinputlisting[language=z80]{src/code/z80/crt0.s}

The first thing we request from the Z-80 is to jump to label \icode{init} which we placed at \icode{0x100}. This is done to leave space for the interrupt handling routine.

\icode{init} is simple. We setup the stack pointer, enable interrupt mode 1 and request the first interrupt. Note that all symbols provided by C functiosn are prefixed with underscore (\icode{\_}) and studied later.

\subsection{Z-80 interrupt}
The Z-80 can work in interrupt modes 0, 1, and 2. Modes 0 and 2 are the most powerful and the most complex since they imply retrieving the id of the peripheral requesting the interrupt by reading a byte on the data bus. This mecanim allows for many devices to get the attention of the Z-80 but in the case of the CPS-1 it is overkill. 

Only one peripheral generates an interrupt (the YM2151 timer). For this case the Z-80 has a perfect interrupt mode 1 where if the \icode{INT} line is asserted, the CPU jumps to address \icode{0x38}. This explains why we have placed our interrupt handler at \icode{.org 0x38}.


\lstinputlisting[language=C]{src/code/z80/interrupts.c}

In function \icode{schedInterrupt} we write to the YM2151 counter register a value \icode{XXXXX}. Since the music chip is running at 3.58 MHz this result in an interrupt generate 3,580,000 x XXX = 4ms later. Notice \icode{\_\_att} which allows to manually place variables.

With this bootstrap, we have effecively implemented the system we envisioned in the "Theory" chapter with two "threads" running concurrently (but not in parallel).

\subsection{Initializing variables}
Before we can jump to \icode{main} function, the bootstrap needs to initalize C variables. As we saw earlier, initialized writable variables use RAM address but intiial values are stored in ROM. These must be copied.

\lstinputlisting[language=z80]{src/code/z80/initVar.s}

\lstinputlisting[language=z80]{src/code/z80/copyvar.s}
 

No initialization of BSS

Also explain \icode{s\_} and \icode{l\_}

\subsection{Linker script}

The last piece of the memory map puzzle is to write a linker script to have \icode{\_CODE} and \icode{\_DATA} areas in the right location.

\lstinputlisting[]{src/code/z80/main.lk}

Compiling is done via XXXX. See \icode{ccps} in verbose mode to see individual commands issued to \icode{sdcc} compiler and \icode{sdldz80} linker.

Trick: Always check out the linker .map file to make sure things are were you wanted them.

\subsection{Latches Circular buffer}

Tell about variation in Z-80 driver. Final fight uses the same sound id whereas SF2 uses a translation layer.


\subsection{Sound Effects with MSM6295}
Bla

\subsection{Music with YM2151}
Bla
