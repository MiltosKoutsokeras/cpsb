\chapter{Introduction} 

The origin of Capcom can be traced back to the foundation of two companies by Kenzo Tsujimoto, I.R.M. Corporation in 1979 and its subsidiary Japan Capsule Computers Co in 1981. Located in Osaka prefecture, the two companies respectively manufactured and distributed electrical games.

After a merging in 1981 the resulting entity, Sanbi, was rebranded "Capcom" in 1983. The first medal cabinet, released the same year, was a base-ball title named "Little League".

The nickname of their products, "\textbf{Cap}sule \textbf{Com}puters", summarized the values of the company. Intending to extend beyond the trendy personal computers of the era, the "coin-ops" were sold as "capsules packed to the brim with gaming fun". The hard outer shell embodied the desire to protect intellectual property and prevents illegal copies which resulted in inferior imitations.


 In 1984, Capcom entered the world of video-games with their first title "Vulgus".
Arcades were a competitive world where lined up cabinets had only a few seconds to catch the eyes of a customer. It was especially difficult to compete for a company which, at the time, did not have the best technology.

\begin{q}{Noritaka Funamizu ("Poo"), Capcom Game Planner (a.k.a Director\cite{planner})}
I always considered Capcom as someone fighting with a bamboo stick. We didn't have the resources to equal Sega's or Namco'
s hardware. 

While they were racing in F1 cars, we were driving Hondas.
\end{q}


\begin{figure}[H]
\img{1943.png}
\caption*{1943: The Battle of Midway by Capcom (1987)}
\end{figure}

The metaphor appears justified when comparing side by side two titles from 1987, Capcom's "1943" (which Poo directed) and Sega's "Afterburner".

Sega's platform named "X Board", was on a league of its own. Its core ran on a dual 12.5 MHz Motorola 68000 CPUs. The graphic processor, Sega Super Scaler chipset clocked at 50MHz, was capable of both scaling and rotating up to 256 sprites over two background layers, one "road" layer, and even had raycasting capabilities. 

Its sound system, ran by a 4MHz z80 sported a SegaPCM 16-channel stereo chip capable of rendering digitized sound effects far surpassing what could be achieved with FM synthesis.

On the opposing corner, Capcom's valiant board featured a 6Mhz z80 CPU with a graphic system capable of animating one sprites layer on top of one text layer, and two background layers. 

The sound system, piloted via a second z80 running at 3Mhz generated both musics and sound effect via Yamaha FM synthesis.




\begin{figure}[H]
\img{afterburner.png}
\caption*{Afterburner by SEGA (1987)}
\end{figure}

% "The Battle of Midway" was no match to Sega's blockbuster 
% but 
Despite its simple graphics, 1943 sold honorably. It even managed to become the second highest-grossing table arcade game of 1987 thanks to its engaging gameplay. 

The dawn of Capcom history saw more titles managed doing well despite their limited technology. In 1985, "Wolf of the BattleField" a.k.a Commando\index{Game!Commando}, conquered the world and in particular UK where test locations resulted in orders of a thousand units\cite{cgm4}.

\begin{figure}[H]
\img{commando.png}
\caption*{Commando (1985)}
\end{figure}

USA distributor, Data East, prominently advertised the profit prospects of the cabinet.
\vfill

\img{commando_cabinet.png}


Released in 1984, Ghost'n Goblins\index{Game!Ghost'n Goblins} is another video-games which is emblematic of Capcom's capacity to do more with less in its early days. 

With the same "bamboo-stick" technology used by Commando, and the same Planner Tokuro Fujiwara directing a small team made of three, programmer Toshio Arima, artist Masayoshi Kurokawa, and composer Ayako Mori, Ghost'n Goblins was another hit.

In this medieval fantasy inspired title, the hero is to rescue an abducted lover. Arthur had to face hordes of Zombies, Magicians, Skeletons, Red Arremers, Flying Knights, and the most unforgiving control system known to mankind.

The story was well put together. Bosses such as Unicorns, Dragons, Satan, and the chief Astaroth were well-animated. Despite its nightmare inducing difficulty, player liked the game. 

% During its early years, a pattern in Capcom titles emerged. Teams could not rely on their technology to make good games. What they lacked in raw power they had to made it up with imagination, tinkering, and heart.
\vfill 

\begin{figure}[H]
\img{gb.png}
\caption*{Ghost’n Goblins (1984)}
\end{figure}
\pagebreak

It became the 10\textsuperscript{th} highest-grossing arcade game in Japan and took the 9\textsuperscript{th} spot in US.
\vfill 
\begin{figure}[H]
\img{gb_flyer.png}
\caption*{Ghost’n Goblins flyer (1984)}
\end{figure}
\pagebreak

What Capcom lacked in raw power, they had to made it up with imagination and tinkering. But wit was not always enough. In 1987, Capcom published "Street Fighter", an audaciously innovative title but a commercial failure.

In "Street Fighter" players controlled their character with a standard joystick but used two large pneumatic pushers for kicks and for punches. Pipes conducted air to the board where pressure was measured. The harder a player punched the button and the higher the damage inflicted.

Without tactile feedback, players tended to smash as hard as possible, forgetting to manage their effort. After a few rounds the right arm was shot and fatigue made rounds not fun. Besides enjoyment, there were also the issues of injuries and tendinitis. To rectify, Capcom retrofitted the control system with something more "standard" while still allowing player to select the power via six "normal" buttons. 

The game became playable but resulted in unimpressive graphics and sluggish controls. The cabinet was largely ignored and despite attempt to boost sales with discounts, "the most amazing dedicated upright ever" soon fell into oblivion.


\begin{figure}[H]
\img{sf1_4_3.png}
\caption*{Street Fighter 1 (1987)}
\end{figure}

% Announced as as sequel, it would have been fair for players to expect mere improvements from the next iteration. What came out instead was to stir up players passions. Staring at a screenshots of each games give a good idea of how much of a world apart these two titles were.


\begin{figure}[H]
\img{sf1_cabinet.png}
\caption*{Street Fighter 1 cabinet flyer}
\end{figure}


\section{Costly Production}
Besides facing technologically superior competitors, Capcom had to deal with a constantly evolving production pipeline. Looking at the Printed Circuit Boards\index{PCB!Printed Circuit Boards} (PCBs)\index{PCB} hosting their games from 1984 to 1988 reveals a high variations of components.

Summarizing Capcom's usage of the Motorola 6809, Zilog Z80, Motorola 68000, Intel 8751 (MCU), YM2203, YM2151, YM2149, and MSM5205 in a table shows that even titles produced the same year wouldn't necessarily feature the same chips. 

Even the ubiquitous Z-80 was used inconsistently since it could be dedicated to game logic, audio or both like in "1942" board which features two of them.



\begin{figure}[H]
{ 
\setlength{\tabcolsep}{3.0pt}
\setlength\cmidrulewidth{\heavyrulewidth} % Make cmidrule = 
\begin{tabularx}{\textwidth}{Xccccccccc}

  & & \multicolumn{4}{c}{CPU} &  \multicolumn{4}{c}{SOUNDS} \\
  \cmidrule(lr){3-6}
  \cmidrule(lr){7-10}
 
  \textbf{Game Name} & \textbf{Year} & \textbf{M6809} & \textbf{Z80} & \textbf{M68000} & \textbf{i8751} & \textbf{2203} & \textbf{2151} & \textbf{2149} & \textbf{5205} \\               
  % Game & Year & 6809 & z80 & 68k & MCU & 2203 & 2151 & 2149 & 5205 \\               
  \toprule    
\href{https://www.youtube.com/watch?v=45ELzG1ivEA}{Vulgus}
                & 1984          &               &      X       &              &              &               &               &       X       &               \\
\href{https://www.youtube.com/watch?v=R5mg6XPqtBs}{Higemaru}
                & 1984          &               &      X       &              &              &               &               &       X       &               \\
\href{https://www.youtube.com/watch?v=Em7UwOOBvlA}{1942}
                & 1984          &               &      X       &              &              &               &               &       X       &               \\
  \toprule    
\href{https://www.youtube.com/watch?v=1qctKI_t5eY}{Commando}
                & 1985          &               &      X       &              &              &       X       &               &       X       &               \\
\href{https://www.youtube.com/watch?v=SugLAqaPhqA}{Ghost'n Goblins}  
                & 1985          &       X       &      X       &              &              &       X       &               &       X       &               \\
\href{https://www.youtube.com/watch?v=mrO9qwGXdy8}{Gun Smoke}        
                & 1985          &               &      X       &              &              &       X       &               &       X       &               \\
\href{https://www.youtube.com/watch?v=cIC2mNNryZg}{Section Z}
                & 1985          &               &      X       &              &              &       X       &               &       X       &               \\
  \toprule    
\href{https://www.youtube.com/watch?v=L1FVWdlQNG8}{Trojan}
                & 1986          &               &      X       &              &              &       X       &               &               &       X       \\
\href{https://www.youtube.com/watch?v=57lg9pFUgco}{Speed Rumbler}
                & 1986          &       X       &      X       &              &              &       X       &               &       X       &               \\
\href{https://www.youtube.com/watch?v=0QyLx94PMio}{Dyn Side Arms}
                & 1986          &               &      X       &              &              &       X       &               &       X       &               \\
\href{https://www.youtube.com/watch?v=0f4jWQyf-fs}{Legendary Wings}
                & 1986          &               &      X       &              &              &       X       &               &       X       &               \\
  \toprule    
\href{https://www.youtube.com/watch?v=kntCwchJWfw}{1943}
                & 1987          &               &      X       &              &              &       X       &               &       X       &               \\
\href{https://www.youtube.com/watch?v=ZzKStmMAiHM}{Black Tiger}
                & 1987          &               &      X       &              &      X       &       X       &               &       X       &               \\
\href{https://www.youtube.com/watch?v=kVLCv-YgWco}{Street Fighter}
                & 1987          &               &      X       &      X       &      X       &               &       X       &               &       X       \\
\href{https://www.youtube.com/watch?v=1ZtwOGN-ZeE}{Tiger Road}
                & 1987          &               &      X       &      X       &              &       X       &               &       X       &       X       \\
  \toprule    
\href{https://www.youtube.com/watch?v=zG620nr7vko}{Bionic Commando}
                & 1988          &               &      X       &      X       &      X       &               &       X       &               &               \\
\href{https://www.youtube.com/watch?v=zG620nr7vko}{F1-Dream}
                & 1988          &               &      X       &      X       &      X       &       X       &               &       X       &               \\
  \toprule    
\end{tabularx}%
}\caption*{Usage of chips in Capcom arcades from 1984 to 1988\cite{cps0chipslist}.}
\label{fig:capcom_pcbs}
\end{figure}


Even though most games did mostly the same thing (move sprites over backgrounds), the hardware had to be re-invented over and over again. 

The pipeline also slowed down production since game programming pace was hindered by bugs in the hardware. Full speed could only be achieved in the later stage of development which placed Capcom at a further disadvantage with its competitors.





\section{Plagued by piracy}
Since its PCBs were made with of-the-shelf components, counterfeiters where able to physically copy them, dump the software ROMs and build replicas called "bootlegs". 

Without having to offset the cost of development, these copies sold for less than official games and these missed sales weighted heavily on Capcom financial health. 




\section{Capcom NT (New Technology)}
Production difficulties, competitions, and piracy painted an uncertain future for Capcom's arcade division. As history would have it, not only they survived, they thrived.

A new era began in 1988. With the release of Forgotten Worlds and Strider, players got a first glimpse into the kind of games the Osaka company was now capable of.  



\label{nin_fw}
\begin{figure}[H]
\img{fw_4_3.png}
\caption*{Forgotten Worlds (1988)}
\end{figure}



 
\label{fw_flyer}
  \begin{figure}[H]
  \img{cps1_announcement.jpg}
  \caption*{Capcom's CPS-1 announcement flyer (1989).}
  \end{figure}

With both games using a mysterious and intriguing "new technology" (later renamed \textbf{CP-System} and then \textbf{CPS-1}), Capcom production quality improved. Not only massive sprites moved on the screen, they were made of many more colors and floated on top of several layers simulating parallax. Strider had impressive slopped surfaces which the hero could climb. The audio was better, featuring digitized sounds.

The first mega-hit came with "Final Fight"\index{Game!Final Fight} in 1989. Up to that point in time, the "beat 'em up" genre had been dominated by Technos thanks to its excellent series of Kunio-kun (known outside of Japan as "Renegade") and mega-hit Double Dragon. 

With Cody, Guy, and Mike, Capcom cleared the room. Despite a minuscule budget of 2MiB for the graphics, the art team, let by Akira Yasuda (a.k.a Akiman) used the full capability of the CPS-1 to provide gorgeous visuals and engaging musics. The game-play was fantastic. And more importantly the timing was right for the US market where "beat 'em up" were all the rage.



  

\label{nin_ff}
 \begin{figure}[H]
\img{ff_4_3.png}
\caption*{Final Fight (1989)}
\end{figure}

"Final Fight" soon became Capcom's top-selling game at the time\cite{birth_of_chunli} and established the company as an unarguable arcades powerhouse.

\pagebreak

Perhaps the best compliment came from competitors who, years later, would admit of the demoralizing effect "Final Fight" had on them.

\begin{q}{Yoshihisa Kishimoto, Planner (Double Dragon \& Kunio-kun)\cite{dd} }
  

  % Without a doubt, 
  The people from Capcom hurt Technos Japan a lot with Final Fight which was superior on every level to Double Dragon III. Not only they had amazing designers, they also gave their teams means to innovate on the hardware level (CPS-1). 

  For us it was an horrible awakening because it proved we had been unable to evolve as fast as them.
  %<ref src="">Yoshihisa Kishimoto, Enter The Double Dragon: p142</ref>
  \end{q}



Not only quality improved, quantity also increased. Thanks to its stable platform and tools, Capcom was able to release more than thirty tile between 1988 and 1995, all based on its CPS-1 platform. 

And then in 1991 "Street Fighter 2"\index{Game!Street Fighter 2} took the world over by storm.

\label{nin_sf2}
\begin{figure}[H]
\img{sf2_4_3.png}
\caption*{Street Fighter 2 (1991).}
\end{figure}

\index{Game!All from 1988 to 1995}
\begin{figure}[H]
{ \setlength{\tabcolsep}{3.0pt}
\begin{tabularx}{\textwidth}{Xrr} 
  \textbf{Game Name} & \textbf{ GFX }  & \textbf{ Year } \\               
  \toprule    
\href{}{Forgotten Worlds} & 4 MiB & 1988 \\ 
\href{}{Ghouls'n Ghosts} & 3 MiB & 1988 \\ 
  \toprule    
\href{}{Strider} & 4 MiB & 1989 \\ 
\href{}{Dynasty Wars} & 8 MiB & 1989 \\ 
\href{}{Willow} & 4 MiB & 1989 \\ 
\href{}{U.N Squadron} & 2 MiB & 1989 \\ 
\href{}{Final Fight} & 2 MiB & 1989 \\ 
  \toprule    
\href{}{1941: Counter Attack} & 2 MiB &  1990 \\ 
\href{}{Mercs} &  3 MiB & 1990 \\ 
\href{}{Mega Twins} & 2 MiB & 1990 \\ 
\href{}{Magic Sword} & 2 MiB & 1990 \\ 
\href{}{Carrier Air Wing} & 2 MiB  & 1990 \\ 
\href{}{Nemo} & 2 MiB &  1990 \\ 
  \toprule    
\href{}{Street Fighter II: The World Warrior} & 6 MiB & 1991 \\ 
\href{}{Three Wonders} & 4 MiB & 1991 \\ 
\href{}{The King of Dragons} & 4 MiB & 1991 \\ 
\href{}{Captain Commando} &  4 MiB & 1991 \\ 
\href{}{Knights of the Round} & 4 MiB  & 1991 \\ 
  \toprule    
\href{}{Street Fighter II: Champion Edition} & 6 MiB & 1992 \\ 
\href{}{Adventure Quiz: Capcom World 2} & 2 MiB & 1992 \\ 
\href{}{Varth: Operation Thunderstorm} & 2 MiB & 1992 \\ 
\href{}{Quiz \& Dragons: Capcom Quiz Game} & 2 MiB & 1992 \\ 
\href{}{Street Fighter II' Turbo: Hyper Fighting} &  6 MiB & 1992 \\ 
  \toprule    
\href{}{Ken Sei Mogura: Street Fighter II} & 6 MiB & 1993 \\ 
\href{}{Pnickies} & 2 MiB & 1993 \\ 
  \toprule    
\href{}{Quiz Tonosama no Yabo 2} &  4 MiB & 1995 \\ 
\href{}{Pang! 3} & 2 MiB  & 1995 \\ 
Mega Man the Power Battle & 8 MiB  & 1995 \\

\toprule    
\end{tabularx}%
}\caption*{Capcom CPS-1 based arcade games from 1988 to 1995.}
\end{figure}







More than a game, "Street Fighter 2" was phenomenon. Sitting at the intersection of Capcom's new found technology and the artistic and design skills its staff had learned in the necessity of survival.

It was quality quantum leap which blew gamers minds. The eight characters came in different size, shape and gender. They each had their own moves and special abilities. These characterization gave them depth. The music was engaging, the audio effects crisp. Control felt precise. 

The video was breathtaking with per-line parallax running at 60Hz showcasing Akiman's team artistic talent. 


\begin{figure}[H]
\img{sf2_tournament.png}
\caption*{Street Fighter 2 tournament flyer.}
\end{figure}

The game immediately developed a cult following. Players had to wait in lines to drop a quarter. Purchasing "Continue" with quarter was frowned upon by other players impatiently waiting for their turn. Operators purchased multiple copies to reduce the wait time but at the end of the days machines still overflowed with coins\cite{sf2_oral_history}.

The popularity was such that tournaments with handsome rewards (opposing page) were held. 

By 1995, the series had generated \$2.3 billion with 200,000 cabinets sold\cite{usgamer20160101} (60,000 World Warriors units and 140,000 Champion Edition units). By 2017 that figure reached \$10.61 billions\cite{gamerevolution20140126}, making "Street Fighter 2" the third top grossing game of all time.

\begin{trivia}
Who is this blinking \textbf{NIN} featured in the high-score of Forgotten Worlds (p\pageref{nin_fw}), Final Fight (p\pageref{nin_ff}), and Street Fighter II (p\pageref{nin_sf2})? It is the nickname of Akira Nishitani, the planner (designer) on all these titles!
\end{trivia}

\section{Ode to CP-System}

This book is a engineering love letter to the system that enabled Capcom to operate its metamorphosis, to evolve from a company fighting for survival to become an arcades household.

The goal of this work is to understand the CP-System completely. This will (hopefully) be achieved by first exposing the hardware and then progressively moving up, all the way to the programming and game engine architecture level.

In the first chapter, the hardware of the four systems making the CP-System is explored.
\begin{itemize}[topsep=0pt]
\item Control System.
\item Audio System.
\item GFX System.
\item Video System.
\end{itemize}
 Beyond the hard reality of silicon and bus lines, a discussion of the design choices and real-life examples of how games leveraged features is provided when relevant.


The sub-sequent chapters study the software and how to build it. In particular are explained the four ROM groups resulting in a game. 

\begin{itemize}[topsep=0pt]
\item Motorola 68000 ROM.
\item Zilog Z80 programming and YM2151 (musics) ROM.
\item MSM6295 (audio samples) ROM.
\item CPS-A/CPS-B (GFX) ROM.
\end{itemize}

\begin{CJK}{UTF8}{min}
Ready? さあ行こう!
\end{CJK}



