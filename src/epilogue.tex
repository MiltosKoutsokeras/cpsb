\chapter{Epilogue} 

Spending a year studying the CPS-1 was a passion project leading to an epiphany.

Capcom's hardware was instrumental in the company survival. It did a great job
in  fighting piracy and streamline production. But if we look at the visual effects and audio it could
output, and compare it to what competitors lined up, it was not an outstanding system.

The difference is especially striking if you compare the CPS-1 to the flagship of Capcom's
main rival, the Neo-Geo by SNK. 

Here we find sprite-shrinking (extensively used in Super Sidekicks),
auto-animation (a feature allowing to define all frames in an animation and to fire-forget), and
the capability to detect HSYNC to implement gorgeous raster effects. And what to say about the  700+ megabits capacity?

As I discovered the CPS-1 could do none of these things, I also polled my memories and realized 
I never noticed a difference as a player.

My take away from this study is that technology is a very beautiful vector. It is was the first 50\% of
Capcom success story. But it was not all of it.

The element that took Capcom cabinets to another level were the artist and designers Capcom hired and nurtured. 
They put a soul in the machine. 

And lucky us to have been there to enjoy it.

