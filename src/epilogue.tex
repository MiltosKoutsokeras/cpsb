\chapter{Epilogue} 

The CPS-1 study was a passion project that took over a year to complete in my spare time. As a technology inclined person, the initial goal was to obsessively explore the hardware in order to understand it down to the metal.

Discovering the internals of Capcom's machine was fascinating and borderline addictive. I often found myself in the wee-hours exploring schematics or experimenting with code. The technology that unraveled under my eyes confirmed the key part it played in shaping Capcom's destiny. 

The CPS-1 did a fantastic job at fighting piracy and streamline production. But if its tile engine was a solid and flexible solution to developers' problems, the comparison with other system puzzled me.

Having studied the evolution of 3D game engines in the 90s I was biased towards the importance of visuals superiority. How could I justify the success of the CPS-1 against the Neo-Geo? 

On paper, SNK's machine could do more than Capcom's. It could scale via sprite-shrinking (extensively used in Super Sidekicks),
had auto-animation (a feature allowing to define all frames in an animation and to fire-forget found in profusion in Metal Slug), and the capability to detect HSYNC to implement gorgeous raster effects. And what to say about the 700+ megabits capacity?

Writing this book made it abundantly clear that, past a certain point, technology cease not matter. Technology is a vector which, when it is well executed, will bring a product 50\% toward the finish line. But it is only a starting point.

As my exploration neared its end, I found myself admiring more and more the work of the people who breed life into the silicon.   
Courageously tracking allocations with paper and scissors, they did not even have a mouse to digitize their drawing. Their creativity, devotion, and pragmatism were the other 50\% needed to ship great games.

The CP-System was a marvel which brought important technological advantage to the table. But it is also thanks to the devotion of its artists and designers that Capcom ended up with a soul in its machine.

Lucky us.

- Fabien Sanglard
