\chapter{Epilogue} 

The CPS-1 study was a passion project that took over a year to complete in my spare time. The goal was to obsessively explore the hardware, understand it down to the metal, and learn how to program it. As it is often the case, the journey took an unexpected turn and I came out of the adventure with more than I initially anticipated.

In the beginning, discovering the internals of Capcom's machine was fascinating and borderline addictive. I often found myself in the wee-hours exploring schematics or experimenting with code. The technology that unraveled confirmed the key part it played in shaping Capcom's destiny. 

It is when I starting studying the systems competing against the CP-System that my opinion started to evolve. 

Capcom's arch-nemesis, SNK, had built an impressive machine which surpassed the CP-System. Games were built relying exclusively on sprites without using limiting tilemaps. While the CPS-1 could display 256 sprites, the Neo-Geo could achieve 381. Each of the Neo-Geo sprites could be scaled via a shrinking technique extensively used in successful titles such as Super Sidekicks. 

The list of features goes on. Auto-animation allowed to defining and forgeting an animation, a feature used profusely in Metal Slug for the gorgeous result that made it famous. HSYNC detection unlocked raster effects. The 330 megabits capacity of its boards was proudly advertised.

Yet, despite hardware's shortcomings, Capcom games were able to hold their own. In several occurrences some even managed to achieve much greater success than titles running on the Neo-Geo. It was as if, past a certain point, technology did not matter that much.

As this book was coming to an end, I found myself admiring more and more the work of the people who breed life into the silicon. Yes, they had a good platform to work with but it was not a silver bullet either. These creatives slept under their desk. They courageously tracked allocations with paper and scissors, they entered pixel colors by hand, tile by tile, using a keyboard. They worked long nights and passed ROM chips using string though the windows in order to meet deadlines.

This venture started with the goal of giving readers a greater appreciation for the hardware. It ends with an author having opened his eyes to the artists and designers who put a soul in the machine.

- Fabien Sanglard
