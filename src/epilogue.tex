\chapter{Epilogue} 

The CPS-1 study was a passion project that took me over a year to complete. As a technology inclined person, the goal was to obsessively explore the machine in order to understand it down to the metal.

Discovering the internals of Capcom's hardware was fascinating and borderline addictive. I often found myself in the wee-hours exploring schematics or experimenting with code. The technology that unraveled under my eyes made it undeniable that it played a key part in crafting Capcom's success. 

The CPS-1 did a fantastic job at fighting piracy and streamline production. But if its tile engine was a solid and flexible solution to developers' problems, the comparison with other system puzzled me.

Having studied the evolution of 3D game engines in the 90s I was biased towards the importance of visuals superiority. How could I justify the success of the CPS-1 against the Neo-Geo? 

On paper, SNK's machine could do more than Capcom's. It could scale via sprite-shrinking (extensively used in Super Sidekicks),
had auto-animation (a feature allowing to define all frames in an animation and to fire-forget found in profusion in Metal Slug), and the capability to detect HSYNC to implement gorgeous raster effects. And what to say about the 700+ megabits capacity?

The answer to my question is that, past a certain point, technology cease not matter. Technology is a vector which when is well executed will bring a product 50\% toward the finish line. But it is not all of it.

As my exporation neared its end, I found myself admiring more and more the work of the people who breath life into the silicon.   

There is a reason Capcom games were able to compete against superior hardware. It is the same reason why, thirty years after the last CP-S left the factory, player play these technologically obsolete games.

We still play them because these games have a mind of their own. Thanks to the devotion of its artists and designers, there is a soul in these machines.

- Fabien Sanglard
